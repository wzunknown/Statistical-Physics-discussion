% \documentclass[aspectratio=169]{wzbeamer}
% \usepackage[hyperref=true,style=numeric,backend=biber,sorting=nty,autocite=inline,backref=true,urldate=iso,date=iso,seconds=true]{biblatex}
% \addbibresource{ref.bib}
% \usepackage{animate}
% \usetheme[progressbar=frametitle, numbering=counter]{metropolis}
% \usepackage{fontspec}
% \setmainfont{Times New Roman}
% \setsansfont{Arial}
% \setmonofont{Cambria}

% \usepackage{etoolbox} % for '\AtBeginEnvironment' macro
% \AtBeginEnvironment{pmatrix}{\everymath{\displaystyle}}

% \graphicspath{{./figs/}}

% % \newcommand{\Pr}{\text{Pr}}
% \begin{document}
%++++++++++++++++++++++++++++++++++++++++++++++++++++++++++++++++++++++++++++++++++++++++++++++++++++++++++++++++++++++++++++++++++++++++++++++++++++++++++++++++++++++++++++++++++++++++++++++++++++++++++++++++++++++++++++++++++++++++++++++++++++++++++
\section{Jarzynski Equality}
    %==========================================================
    \subsection{Introduction}
    \begin{frame}{Jarzynski \& Crooks}
        \hspace{.05\textwidth}
        \begin{figure}
            \centering
            \subfigure[Christopher Jarzynski]{
                \includegraphics[width=.3\textwidth]{jarz.jpg}
            }
            \vspace{.05\textwidth}
            \subfigure[Gavin E. Crooks]{
                \includegraphics[width=.5\textwidth]{Gavin_Crooks.jpg}
            }
        \end{figure}
    \end{frame}
    \begin{frame}{正则系综与自由能}
        当系统与一个温度为$T$的大热库有热接触时, 平衡态系综的微观态分布满足, 
        \begin{equation}
            \rho(\vb*x) = \ee^{\beta(F - H(\vb*x))}, \quad \int_\Gamma\dd\vb*x\ \rho(\vb*x) = 1,\quad \beta = \frac1{k_B T}
        \end{equation}
        $F$为自由能, 由归一化决定,
        \begin{equation}
            \ee^{-\beta F} = \int_\Gamma\dd\vb*x\ \ee^{- \beta H(\vb*x)}
        \end{equation}
        更加普适的定义是,
        \begin{equation}
            F = \expval{H} - T S,\quad S = -\int_\Gamma\dd\vb*x\ \rho(\vb*x)\ln \rho(\vb*x)
        \end{equation}
    \end{frame}
    \begin{frame}{热力学第二定律}
        \begin{columns}
            \column{.5\textwidth}
            热力学第二定律,
            \begin{equation}
                \dbar Q \leqslant T\dd S 
            \end{equation}
            当系统和大热库始终处于热平衡的情况下, 从$A\to B$,
            \begin{equation}
                F_B - F_A = \Delta F \leqslant W^F 
            \end{equation}
            统计解释,
            \begin{equation}
                \Delta F \leqslant \expval{W^F} = \int W\rho^F(W)\dd W
            \end{equation}
            \column{.5\textwidth}
            \begin{figure}
                \centering
                \includegraphics[width=\textwidth]{para.png}
                \caption{哈密顿量参数空间}
            \end{figure}
        \end{columns}
    \end{frame}
    \begin{frame}{热力学第二定理}
        \begin{columns}
            \column{.5\textwidth}
            我们考虑一下逆过程, $B\to A$,
            \begin{equation}
                F_A - F_B \leqslant \expval{W^R} = \int W\rho^R(W)\dd W
            \end{equation}
            把两个过程放在一起观察,
            \begin{equation}
                - \expval{W^R} \leqslant \Delta F \leqslant \expval{W^F}
            \end{equation}
            \pause
            \column{.5\textwidth}
            \begin{figure}
                \centering
                \includegraphics[width=\textwidth]{pw.png}
                \caption{概率分布}
            \end{figure}
        \end{columns}
    \end{frame}
    \begin{frame}{主要结论}
        \begin{itemize}
            \item Jarzynski Equality(JE):
            \begin{equation}
                \expval{\ee^{-\beta W}} = \ee^{-\beta \Delta F}
            \end{equation}
            \item Crooks Fluctuation Theorrem(CFT):
            \begin{equation}
                \frac{P^F(\omega)}{P^R(-\omega)} = \ee^{\omega}\quad\Rightarrow\quad \frac{\rho^F(W)}{\rho^R(-W)} = \ee^{\beta(W - \Delta F)}
            \end{equation}
        \end{itemize}
    \end{frame}
    %==========================================================
    \subsection{Jarzynski Equality(JE)}
    \begin{frame}{JE}
        \begin{alertblock}{Jarzynski Equality}
            将一个与温度为$T$的大热库热接触的系统, 改变系统的宏观参数, 从$A\to B$, $A$和$B$都是平衡态, 对于各种可能的做功, 我们有,
            \begin{equation}
                \expval{\ee^{-\beta W}} = \ee^{-\beta \Delta F}
            \end{equation}
        \end{alertblock}   
    \end{frame}
    \begin{frame}{证明}
        Jarzynski在1997年的PRL中实际上只给出了一种特殊情形下的证明, 我们考虑一个系统, 刚开始时和大热库处于热平衡, 满足正则系综, 然后进行控制参数的改变, 改变过程中和热库之间没有热量交换, 也无需处于平衡态, 结束之后, 保持控制参数不变, 系统和热库接触, 回到温度$T$. 
        
        哈密顿量$H = H(\vb*x,\lambda)$, 对于每一个参数$\lambda$, 都可以得到一个自由能$F_\lambda$,
        \begin{equation}
            \ee^{-\beta F_\lambda} = \int\dd\vb*x\ \ee^{-\beta H(\vb*x, \lambda)}
        \end{equation}
    \end{frame}
    \begin{frame}{证明}
        系统哈密顿量变化,
        \begin{equation}
            \dd H = \dd Q + \dd W \quad\Leftrightarrow\quad \dd{H(\vb*x(t),\lambda(t))} = \dd\vb*x\pdv{H}{\vb*x}  + \dd\lambda\pdv{H}{\lambda}
        \end{equation}
        可以知道做功$W$,
        \begin{equation}
            W = \int_0^{t_s} \dd t\ \dot\lambda\pdv{H}{\lambda} = H(\vb*x(t_s),B) - H(\vb*x(0),A)
        \end{equation}
        $W$与路径无关, $W = W(\vb*x(0))$,
        \begin{equation}
            \expval{\ee^{-\beta W}} = \int \dd\vb*x(0)\ \frac{\ee^{-\beta H(\vb*x(0),A)}}{Z_A} \cdot \ee^{H(\vb*x(t_s),B) - H(\vb*x(0),A)} = \int  \dd\vb*x(0)\ \frac{\ee^{-\beta H(\vb*x(t_s),B)}}{Z_A}
        \end{equation}
    \end{frame}
    \begin{frame}{证明}
        根据Liouville定理,
        \begin{equation*}
            \expval{\ee^{-\beta W}} = \int \dd\vb*x(t_s) \cdot \left|\pdv{\vb*x(t_s)}{\vb*x(0)}\right|^{-1} \cdot \frac{\ee^{-\beta H(\vb*x_(t_s),B)}}{Z_A} = \frac{Z_B}{Z_A}
        \end{equation*}
        \begin{equation}
            \Rightarrow\quad \expval{\ee^{-\beta W}} = \ee^{-\beta \Delta F}
        \end{equation}
    \end{frame}

    %==========================================================
    \subsection{Crooks Fluctuation Theorem(CFT)}
    \begin{frame}{系统演化}
        之前从哈密顿正则方程出发的演化方式, 相当于基于微正则系综. 这种方式的计算显然是有局限性的. 在之前的情形中, 若要考虑热量交换, 就必须考虑相互作用,
        \begin{equation}
            G(\vb*x,\vb*x') = H(\vb*x) + H'(\vb*x') + h_{int}(\vb*x,\vb*x')        
        \end{equation}
        事情变得非常复杂, 我们需要寻找新的角度来考察系统的演化.

        接下来的讨论中, 我们都假设哈密顿量应该是时间反演对称的(比如没有磁场).
    \end{frame}
    \begin{frame}{Markov Chain and Detailed Balance}
        \begin{block}{Markov Chain}
            有一列随机变量$\{X_0,X_1,X_2,\dots\}$, 若满足下式, 则称之为Markov链
            \begin{equation}
                \Pr(X_{n+1}=a_{n+1} | X_{n} = a_n, X_{n-1} = a_{n-1},\dots) = \Pr(X_{n+1}=a_{n+1} | X_{n} = a_n)
            \end{equation}
        \end{block}
        对于离散变量空间和离散时间的情形, 我们只需要知道每一步的转移矩阵$M_n$, 就知道系统的演化,
        \begin{equation}
            M_{n,ij} = \Pr(X_{n+1} = x_j| X_n = x_i),\ \sum_j M_{n,ij} = 1
        \end{equation}
        若$X_n$的概率分布是$\vb*\pi^{(n)} = [\pi^{(n)}_1,\pi^{(n)}_2,\dots]$,那么,
        \begin{equation}
            \vb*\pi^{(n+1)} = \vb*\pi^{(n)}M_n = \vb*\pi^{(0)}M_0\cdots M_{n-1}M_{n}
        \end{equation}
    \end{frame}
    \begin{frame}{Markov Chain and Detailed Balance}
        对于连续变量和连续时间的情形,
        \begin{equation}
            \rho(\vb*x, t) = \int  \rho(\vb*x', t')p(\vb*x',t'; \vb*x, t)\dd\vb*x', \quad \int p(\vb*x',t'; \vb*x, t) \dd\vb*x = 1
        \end{equation}
        如果一个演化方式和初始时间无关, 只和时间差有关, 我们称这样的Markov链是时齐的(time homogeneous).
        \begin{equation}
            p(\vb*x',t'; \vb*x,t) = p(\vb*x'\to\vb*x, \Delta t)
        \end{equation}
    \end{frame}
    \begin{frame}{Markov Chain and Detailed Balance}
        如果有一个分布在演化下保持不变, 这就是平衡态, 他应该满足,
        \begin{equation}
            \rho(\vb*x) = \int p(\vb*x'\to\vb*x, t)\rho(\vb*x')\dd\vb*x',\quad\forall t>0
        \end{equation}
        显然平衡态对于转移概率$p(\vb*x'\to\vb*x,t)$的约束在数学上并不那么强, 我们基于物理的考虑给他赋予一些新的性质.
        \begin{block}{Detailed Balance $\to$ Microscopically reversible}
            \noindent
            \begin{equation}
                \begin{split}
                    \rho(\vb*x)p(\vb*x\to\vb*x',t) &= \rho(\vb*x')p(\vb*x'\to\vb*x,t)\\ \Rightarrow
                    \rho(\hat{\vb*x})p(\hat{\vb*x}\to\hat{\vb*x}',t) &= \rho(\vb*x')p(\vb*x'\to\vb*x,t)\ \text{and}\ \rho(\vb*x) = \rho(\hat{\vb*x})                 
                \end{split}
            \end{equation}
            $\hat{\vb*x}$表示时间反演, i.e.$\ q\to q, p\to -p$.
        \end{block}
        % 时间反演对称性
        % \begin{equation}
        %     \rho(\hat{\vb*x}) = \rho(\vb*x)
        % \end{equation}
    \end{frame}
    \begin{frame}{Markov Chain and Detailed Balance}
        微观可逆性是平衡态的充分条件,
        \begin{equation}
            \int\dd\vb*x'\rho(\vb*x')p(\vb*x'\to\vb*x,t) = \int\dd\vb*x'\rho(\hat{\vb*x})p(\hat{\vb*x}\to\hat{\vb*x}',t) = \rho(\vb*x)
        \end{equation}
        对于不同的情况,
        \begin{table}[H]
            \centering
            \begin{tabular}{cccc}
                \toprule
                &微正则 & 正则 & 特殊($p$,$T$确定)\\
                \midrule
                $\dfrac{p(\hat{\vb*x}\to\hat{\vb*x}',t)}{p(\vb*x'\to\vb*x,t)}$ & 1 &$\ee^{-\beta(H(\vb*x') - H(\vb*x))}$ & $\ee^{-\beta(\Delta H + p \Delta V)}$\\
                \bottomrule
            \end{tabular}
        \end{table}
    \end{frame}
    \begin{frame}{概率密度}
        最后介绍一个求概率密度的公式, 现在有一个变量$x$的概率分布为$f(x)$, 对于一个新的变量$\omega = \omega(x)$, 它满足概率分布$\rho(\omega)$, 
        \begin{equation}
            \rho(\omega) = \int \delta(\omega(x) - \omega) f(x)\dd x
        \end{equation}
        验证,
        \begin{equation*}
            \begin{split}
                \expval{A(\omega)}_\omega &= \int A(\omega) \rho(\omega)\dd \omega\\
                &= \int A(\omega) \delta(\omega(x) - \omega) f(x)\dd\omega\dd x \\
                &= \int A(\omega(x))f(x)\dd x \\
                &= \expval{A(\omega(x))}_x
            \end{split}
        \end{equation*}
    \end{frame}
    \begin{frame}{CFT}
        \begin{alertblock}{Crooks Fluctuation Theorem}
            \begin{equation}
                \frac{P^F(\omega)}{P^R(-\omega)} = \ee^{\omega}
            \end{equation}
            系统和温度为$T$的大热库接触, $\omega$是一个确定参数变化过程$\lambda(t)(t\in[0,\tau])$中整个体系的熵产生, $P^F,P^R$是正过程和逆过程的熵产生分布函数.
        \end{alertblock}
        接下来都取, $k_B = 1$
    \end{frame}
    \begin{frame}{证明}
        对于相空间中某一条路径$\vb*x(t)$, 我们考虑沿这条路径的概率,
        $$
        F: x(0)\xrightarrow{\lambda(t_1)}x(t_1)\xrightarrow{\lambda(t_2)}x(t_2)\cdots \xrightarrow{\lambda(\tau)}x(\tau)
        $$
        $$
        R:  \hat x(0)\xleftarrow{\lambda(t_1)}\hat x(t_1)\xleftarrow{\lambda(t_2)}\hat x(t_2)\cdots \xleftarrow{\lambda(\tau)}\hat x(\tau)
        $$
        \begin{equation}
            \begin{split}
                \mathcal{P}^F[\vb*x(t)]\mathcal D[\vb*x(t)] &\approx\rho^F_0(\vb*x_0) \dd \vb*x_0p_1(\vb*x_0\to\vb*x_1,t_1)\dd\vb*x_1 \cdots p_n(\vb*x_{n-1}\to\vb*x_\tau,\tau-t_{n-1})\dd\vb*x_\tau\\
                \mathcal{P}^R[\hat{\vb*x}(\tau-t)]\mathcal D[\vb*x(t)] &\approx \dd\vb*x_0p_1(\hat{\vb*x}_0\leftarrow\hat{\vb*x}_1,t_1)\dd\vb*x_1 \cdots p_n(\hat{\vb*x}_{n-1}\leftarrow\hat{\vb*x}_\tau,\tau-t_{n-1})\dd\vb*x_\tau\rho^R_\tau(\hat{\vb*x}_\tau)
            \end{split}
        \end{equation}
        $p_i$表示在$\lambda(t_i)$的条件下进行演化. 利用之前的对于细致平衡的讨论,
        \[\dfrac{p(\hat{\vb*x}\to\hat{\vb*x}',t)}{p(\vb*x'\to\vb*x,t)} = \ee^{-\beta(H(\vb*x') - H(\hat{\vb*x}))}
            \]
    \end{frame}
    \begin{frame}{证明}
        可以得到微观可逆性条件,
        \begin{equation}
            \begin{split}
                \frac{\mathcal{P}^F[\vb*x(t)]}{\mathcal{P}^R[\hat{\vb*x}(\tau-t)]} &= \frac{\rho^F_0(\vb*x_0)}{\rho^R_\tau(\hat{\vb*x}_\tau)}\exp(-\beta\sum_i [H(\vb*x_i,\lambda_i) - H(\vb*x_{i-1},\lambda_i)]) \\
                &= \frac{\rho^F_0(\vb*x_0)}{\rho^R_\tau(\hat{\vb*x}_\tau)}\exp(-\beta\int\dd t\ \dot{\vb*x}\pdv{H(\vb*x,\lambda)}{\vb*x})\\
                &= \frac{\rho^F_0(\vb*x_0)}{\rho^R_\tau(\hat{\vb*x}_\tau)}\exp(-\beta Q)
            \end{split}
        \end{equation}
    \end{frame}
    \begin{frame}{证明}
        我们认为$S = \sum_{\vb*x} - \rho(\vb*x)\ln\rho(\vb*x)$, 
        正向熵产生$\omega^F$, 
        \begin{equation}
            \omega^F = \ln\rho^F_0(\vb*x(0)) - \ln\rho^F_\tau(\vb*x(\tau)) - \beta Q
        \end{equation}
        从而,
        \begin{equation}
            \ee^{\omega^F} = \frac{\rho^R_\tau(\hat{\vb*x}(\tau))}{\rho^F_\tau(\vb*x(\tau))}\cdot\frac{\mathcal{P}^F[\vb*x(t)]}{\mathcal{P}^R[\hat{\vb*x}(\tau-t)]} = \frac{\mathcal{P}^F[\vb*x(t)]}{\mathcal{P}^R[\hat{\vb*x}(\tau-t)]}
        \end{equation}
        最后一个等号我们假设有一个时间反演对称性,
        \begin{equation}
            \rho^F(\vb*x) = \rho^R(\hat{\vb*x})
        \end{equation}
        也可以得到$\omega^R = -\omega^F$
    \end{frame}
    \begin{frame}{证明}
        至此, 我们可以求出$P^F(\omega)$,
        \begin{equation}
            \begin{split}
                P^F(\omega) &= \int\delta(\omega-\omega^F) \mathcal P^F[\vb*x(t)]\mathcal D[\vb*x(t)]\\
                &= \ee^{\omega}\int\delta(\omega+\omega^R)\mathcal{P}^R[\hat{\vb*x}(\tau-t)]\mathcal D[\vb*x(t)]\\
                &= \ee^{\omega}P^R(-\omega)
            \end{split}
        \end{equation}
    \end{frame}
    \begin{frame}{CFT$\to$JE}
        对于初末态是正则系综平衡态的情形,
        \begin{equation}
            \rho^F_0(\vb*x) = \ee^{\beta(F_A - H(\vb*x,A))},\quad \rho^F_\tau(\vb*x) = \ee^{\beta(F_B - H(\vb*x,B))}
        \end{equation}
        从而, 熵产生为,
        \begin{equation}
            \begin{split}
                \omega^F &= \ln\rho^F_0(\vb*x(0)) - \ln\rho^F_\tau(\vb*x(\tau)) - \beta Q \\
                &=\beta(H(\vb*x(0),B) - H(\vb*x(\tau),A)-Q) - \beta(F_B-F_A) \\
                &= \beta(W-\Delta F)              
            \end{split}
        \end{equation}
        因此可以知道,
        \begin{equation}
            \frac{P^F(\omega)}{P^R(-\omega)} = \ee^{\omega}\quad\Rightarrow\quad \frac{\rho^F(W)}{\rho^R(-W)} = \ee^{\beta(W - \Delta F)}
        \end{equation}
    \end{frame}
    \begin{frame}{CFT$\to$JE}
        \begin{equation}
            \expval{\ee^{\beta( - W + \Delta F)}}_F = \int\dd W\rho^F(W) \ee^{\beta( - W + \Delta F)} = \int\dd W\rho^R(-W) = 1
        \end{equation}
        \pause
        \includegraphics[width=100pt]{doggy.png}
    \end{frame}
    %==========================================================    
    \subsection{The Second Law of Thermodynamics}
    \begin{frame}{热力学第二定律}
        热力学第二定理统计解释, 
        \begin{equation}
            \expval{\ee^{-\omega}} = \int\dd \omega P^F(\omega) \ee^{-\omega} = \int\dd \omega P^R(-\omega) = 1
        \end{equation}
        利用Jensen不等式,
        \begin{equation}
            1 = \expval{\ee^{-\omega}} \geqslant \ee^{-\expval{\omega}}\quad\Rightarrow\quad \expval{\omega} \geqslant 0
        \end{equation}
    \end{frame}


%++++++++++++++++++++++++++++++++++++++++++++++++++++++++++++++++++++++++++++++++++++++++++++++++++++++++++++++++++++++++++++++++++++++++++++++++++++++++++++++++++++++++++++++++++++++++++++++++++++++++++++++++++++++++++++++++++++++++++++++++++++++++++
\section{Experiment}
    \subsection{JE}
    \begin{frame}{体系选择}
        我们需要验证JE, 利用数学上的相关公式,
        \begin{equation}
            \Delta F = -\frac1\beta \ln \expval{\ee^{-\beta W}} 
            = -\frac1\beta \sum_{n=1}^\infty \frac{(-\beta)^n}{n!} \kappa_n 
            \approx \expval{W} - \frac12\beta\sigma^2
        \end{equation}
        我们发现, 当$\sigma\beta$较大时, $\Delta F$会很大程度上取决于$W$取偏离$\expval{W}$较大的值, 从式子上看, 令$U = \ee^{-\beta W}$,
        \begin{equation}
            \rho_U(U) = \rho_W(W(U)) \left|\dv{W}{U}\right| = \rho_W(-\frac1\beta\ln U) \frac1{\beta U}
        \end{equation}
    \end{frame}
    \begin{frame}{gif}
        \begin{figure}[H]
            \centering
            \animategraphics[width=.8\textwidth]{10}{gif/anim}{0}{39}
            \caption{概率分布的偏离}
        \end{figure}
    \end{frame}
    \begin{frame}{体系选择}
        \begin{columns}
            \column{.7\textwidth}
            我们必须要选择一个介观的体系, 使得$\sigma\beta$不至于太大, 导致实验上误差太大. 
            \pause
            \begin{center}
                {\Large 展开\&折叠RNA!}
            \end{center}
            \column{.3\textwidth}
            \begin{figure}
                \centering
                \includegraphics[width=.8\textwidth]{rna1.png}
                \caption{RNA}
            \end{figure}
        \end{columns}
    \end{frame}
    \begin{frame}{实验对系综的逼近}
        \begin{equation}
                \expval{\ee^{-\beta W}} = \lim_{N\to\infty} \expval{\ee^{-\beta W}}_N 
        \end{equation}
        三种对自由能的估计,
        \begin{equation}
            \begin{split}
                \text{Average work: } & W_A = \expval{W}_N\\
                \text{Fluctuation-dissipation estimate: } & W_{FD} = \expval{W}_N - \frac12\beta\sigma^2\\
                \text{Jarzynski equality: } & W_{JE} = \expval{\ee^{-\beta W}}_N
            \end{split}
        \end{equation}
    \end{frame}
    \begin{frame}{实验设计}
            \begin{figure}
                \centering
                \includegraphics[width=.6\textwidth]{exp2.png}
                \caption{实验设计}
            \end{figure}
    \end{frame}
    \begin{frame}{光镊}
        \begin{figure}
            \centering
            \subfigure[]{
                \includegraphics[height=.4\textwidth]{trap0.png}
            }
            \subfigure[]{
                \includegraphics[height=.4\textwidth]{trap1.png}
            }
            \caption{光镊}
        \end{figure}
    \end{frame}
    \begin{frame}{实验结果}
        \begin{figure}
            \centering
            \includegraphics[height=.4\textwidth]{JE0.png}
            \caption{F-z曲线}
        \end{figure}
    \end{frame}
    \begin{frame}{实验结果}
        \begin{figure}
            \centering
            \includegraphics[height=.25\textwidth]{JE3.png}
            \caption{$W_{\text{diss}} = W - \Delta F$分布}
        \end{figure}
    \end{frame}
    \begin{frame}{实验结果}
        \begin{figure}
            \centering
            \includegraphics[height=.4\textwidth]{JE2.png}
            \caption{三种估计的对比}
        \end{figure}
    \end{frame}
    \subsection{CFT}
    \begin{frame}{CFT的实验验证}
        \begin{figure}
            \centering
            \subfigure[]{
                \includegraphics[height=.35\textwidth]{CFT0.png}
            }
            \subfigure[]{
                \includegraphics[height=.35\textwidth]{CFT1.png}
            }
        \end{figure}
    \end{frame}
    \begin{frame}{CFT的实验验证}
        \begin{figure}
            \centering
            \subfigure[非Gauss分布]{
                \includegraphics[height=.35\textwidth]{CFT2.png}
            }
            \subfigure[与BAR方法(数值模拟)对比]{
                \includegraphics[height=.35\textwidth]{CFT3.png}
            }
        \end{figure}
    \end{frame}
% \end{document}