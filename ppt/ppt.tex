\documentclass[aspectratio=169]{wzbeamer}
\usepackage[hyperref=true,style=numeric,backend=biber,sorting=nty,autocite=inline,backref=true,urldate=iso,date=iso,seconds=true]{biblatex}
\addbibresource{ref.bib}

\usetheme[progressbar=frametitle, numbering=counter]{metropolis}


\title{Liouville Theorem and Jarzynski Equality}
\subtitle{平统讨论班}
\author{王准}
\institute[pku]{北京大学}


\begin{document}
\maketitle 
    
\begin{frame}{Overview}
    \tableofcontents
\end{frame}

\section{Liouville Theorem}
    \begin{frame}{一般情形}
        哈密顿量$H = H(\vb*q, \vb*p) = H(\vb*x)$, $\vb*x = (\vb*q, \vb*p)$
        \begin{equation}
            \left\{
            \begin{split}
                \dot{q}_i &= \pdv{H}{p_i}\\
                \dot{p}_i &= - \pdv{H}{q_i}
            \end{split}
            \right.\quad
            \Rightarrow\quad
            \dot{x}_\alpha = \omega_{\alpha\beta} \pdv{H}{x_\beta}
        \end{equation}
        其中, $i = 1,2,\dots,n, \ \alpha = 1,2,\dots,2n$, 
        \begin{equation}
            [\omega_{\alpha\beta}] = 
            \begin{pmatrix}
                & I \\
                -I & 
            \end{pmatrix}
            \quad\Rightarrow\quad
            \omega_{\alpha\beta} = -\omega_{\beta\alpha}
        \end{equation}
    \end{frame}



%====================================================================================
% \begin{frame}[allowframebreaks]
%     \nocite{*}
% 	\frametitle{References}
% 	\printbibliography[title={References}]
% \end{frame}
\end{document}